\documentclass{article}
\usepackage[utf8]{inputenc}
\usepackage[spanish]{babel}
\usepackage{listings}
\usepackage{graphicx}
\graphicspath{ {images/} }
\usepackage{cite}

\begin{document}

\begin{titlepage}
    \begin{center}
        \vspace*{1cm}
            
        \Huge
        \textbf{Actividad Inicial Informática II}
            
        \vspace{0.5cm}
        \LARGE
         
            
        \vspace{1.5cm}
            
        \textbf{José Alejandro Moreno Mesa}
            
        \vfill
            
        \vspace{0.8cm}
            
        \Large
        Despartamento de Ingeniería Electrónica y Telecomunicaciones\\
        Universidad de Antioquia\\
        Medellín\\
        Marzo de 2021
            
    \end{center}
\end{titlepage}

\tableofcontents
\newpage
\section{Introducción}\label{intro}
En esta actividad se desea llevar dos cartas y una hoja de papel desde una posición inicial, las cartas debajo de la hoja, hasta una posición final, con las cartas en forma de pirámide sobre la hoja, siguiendo una secuencia de pasos específicos escritos previamente. Para esto, se redactaron los pasos, se realizaron tres pruebas a personas distintas y finalmente se obtuvieron las conclusiones.

\section{Pasos para la solución del planteamiento} \label{contenido}
Posición inicial: Dos cartas colocadas una sobre la otra y una hoja de papel cubriendolas.\\\\
Posición final esperada: Ambas cartas sobre la hoja formando una pirámide. 
\begin{enumerate}
\item Levantar la hoja por una de sus esquinas.
\item Colocar la hoja completamente plana al lado de las tarjetas, cuidando de no cubrir ninguna parte de las tarjetas.
\item Levantar ambas tarjetas de la mesa.
\item Poner ambas tarjetas en la misma orientación y una contra la otra.
\item Coger ambas tarjetas desde uno de sus lados más cortos y ponerlas sobre la hoja de manera que que lado más largo de ambas quede perpendicular a la mesa.
\item Sin soltar el lado superior de las tarjetas, separar los lados inferiores de ambas tarjetas que tocan la mesa.
\item Ir aumentando la distancia de los lados inferiores de las tarjetas de a medio centimetro hasta que se sotengan por sí mismas en una posición en la que los lados superiores se toquen y los lados inferiores toquen la mesa.
\item En caso de que se caigan o suelten las cartas, vuelva al paso 3.
\end{enumerate}
\section{Conclusiones}\label{conc}
Con este ejercicio queda clara la importancia de los detalles y el vocabulario empleado a la hora de comunicar una idea o proceso. Cuando mis papás comenzaron a seguir los pasos, me di cuenta que les pareció confuso interpretar la posición en la que quería que colocaran las cartas, por lo que asumo que un lenguaje menos técnico y más coloquial podría haber funcionado mejor.\\\\
Asimismo, es fundamental destacar la importancia de los detalles que se proporcionan en este contexto, puesto que siempre se deben dejar claros los contenidos desde todos los ángulos posibles, incluso cuando se podrían considerar obvios. Sin embargo, es fundamental filtrar la información antes de compartirla, ya que un exceso de detalles puede generar confusión de la que se desea evitar.




\end{document}
